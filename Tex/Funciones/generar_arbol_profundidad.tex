\section{Busqueda en profundidad}
La función \texttt{\textbf{generar\_arbol\_profundidad(M)}} genera una matriz de adyacencia de un árbol generador dada una matriz \texttt{\textbf{M}}, a partir del algoritmo de busqueda por profundidad.
Esta función,  hace uso de una pila auxiliar para conectar un vértice adyacente al anterior.
\begin{lstlisting}[language=python, caption=Función generar\_arbol\_profundidad(M)]
    def generar_arbol_profundidad(M):
        V = set()  # Conjunto de vertices ya conectados
        Vaux = set()  # Conjunto de vertices auxiliar
        vecino = 0  # Vecino mas pequeno
        Pila = []  # Pila auxiliar
        n = len(M)  # Numero de vertices
        MA = [[0 for i in range(n)] for i in range(n)]  # Matriz de adyacencia
        vertices = crea_vertices(M)
        Pila.append(1)
        V.add(1)
        while len(Pila) != 0:  # Acaba cuando la pila esta vacia
            u = Pila[-1]  # Tomamos el ultimo elemento de la pila
            Vaux = set(vertices[u - 1].get("vecinos"))  # Obtenemos el conjunto de vertices no conectados a u
            if len(Vaux - V) > 0:  # Si hay al menos un vecino no conectado a u
                vecino = min(Vaux - V)  # Tomamos el vertice no conectado mas pequeno
                MA[u - 1][vecino - 1] = 1  # Los conectamos
                MA[vecino - 1][u - 1] = 1
                if vecino not in Pila:  # Si el vecino no estaba en la pila, lo agregamos
                    Pila.append(vecino)
                V.add(vecino)  # Agregamos al vecino al conjunto de vertices ya conectados
            else:
                Pila.remove(u)  # Si u no tiene vecinos sin conectar, lo quitamos de la pila
        Pila.clear()
        V.clear()
        Vaux.clear()
        return MA
\end{lstlisting}