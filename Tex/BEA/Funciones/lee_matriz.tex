\section{Entrada}
La función \texttt{\textbf{lee\_matriz(M)}} solicita el número de vértices del grafo, lee una matriz triangular inferior ($n\times n$), es decir las entradas \texttt{\textbf{m$_{i,j}$ con $i\ge j$}}, donde las  entradas \texttt{\textbf{m$_{i,j}$ con $i=j$}} deben ser pares, pues la diagonal representa a los bucles (loops), y luego le asigna el valor de las entradas \texttt{\textbf{m$_{i,j}$}} a las entradas \texttt{\textbf{m$_{j,i}$}}, para asegurar que sea una matriz de adyacencia (debe ser simétrica). Retorna la matriz generada.
\begin{lstlisting}[language=python, caption=Función lee\_matriz()]
def lee_matriz():
    n = int(input("Ingrese el numero de vertices del grafo: "))
    M = [[-1] * n for i in range(n)]
    for i in range(n):
        for j in range(i + 1):
            if i == j:
                M[i][j] = valida_entrada(i, j, True)
            else:
                M[i][j] = valida_entrada(i, j, False)
            if i > j:
                M[j][i] = M[i][j]
    return M
\end{lstlisting}
\subsection{Validación}
La función \texttt{\textbf{valida\_entrada(M)}} valida que los valores leídos sean numéricos y que las entradas de la diagonal sean números pares.
\begin{lstlisting}[language=python, caption=Función valida\_entrada()]
def valida_bucles(i, j, flag: bool) -> int:
    resul: int = 1
    if flag:
        while resul % 2 != 0:
            while True:
                try:
                    resul = int(input(
                        "Ingrese la entrada M[{0}][{1}] (Esta debe ser par): ".format(str(i + 1), str(j + 1))))
                    break
                except:
                    print("No es un valor valido!")
        return resul
    else:
        while True:
            try:
                resul = int(input("Ingrese la entrada M[{0}][{1}]: ".format(str(i + 1), str(j + 1))))
                return resul
            except:
                print("No es un valor valido!")
\end{lstlisting}